\documentclass{beamer}
\usepackage[utf8]{inputenc}
\usepackage[spanish]{babel}
\usepackage{graphicx}
\usepackage{booktabs}

\title{Conclusiones del Marco Teórico}
\author{Tu Nombre}
\institute{Tu Universidad}
\date{\today}

\begin{document}

\frame{\titlepage}

\begin{frame}
\frametitle{Conclusiones}
\begin{itemize}
    \item Los patrones de crimen no son aleatorios, sino influenciados por variables ambientales y socioeconómicas.
    \item Variables como el clima, mes y temperatura afectan significativamente la incidencia delictiva.
    \item La combinación de modelos predictivos avanzados (Redes Neuronales LSTM y Matrices Evolutivas) permite predecir crímenes con alta precisión.
    \item La integración de datos geográficos y temporales mejora la identificación de zonas y momentos de alto riesgo.
    \item Estos modelos facilitan la implementación de estrategias preventivas y optimización de recursos de seguridad.
\end{itemize}
\end{frame}

\end{document}

